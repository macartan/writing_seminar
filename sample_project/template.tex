% Options for packages loaded elsewhere
\PassOptionsToPackage{unicode}{hyperref}
\PassOptionsToPackage{hyphens}{url}
\PassOptionsToPackage{dvipsnames,svgnames,x11names}{xcolor}
%
\documentclass[
  sn-nature,
]{sn-jnl}

\usepackage{amsmath,amssymb}
\usepackage{iftex}
\ifPDFTeX
  \usepackage[T1]{fontenc}
  \usepackage[utf8]{inputenc}
  \usepackage{textcomp} % provide euro and other symbols
\else % if luatex or xetex
  \usepackage{unicode-math}
  \defaultfontfeatures{Scale=MatchLowercase}
  \defaultfontfeatures[\rmfamily]{Ligatures=TeX,Scale=1}
\fi
\usepackage{lmodern}
\ifPDFTeX\else  
    % xetex/luatex font selection
\fi
% Use upquote if available, for straight quotes in verbatim environments
\IfFileExists{upquote.sty}{\usepackage{upquote}}{}
\IfFileExists{microtype.sty}{% use microtype if available
  \usepackage[]{microtype}
  \UseMicrotypeSet[protrusion]{basicmath} % disable protrusion for tt fonts
}{}
\makeatletter
\@ifundefined{KOMAClassName}{% if non-KOMA class
  \IfFileExists{parskip.sty}{%
    \usepackage{parskip}
  }{% else
    \setlength{\parindent}{0pt}
    \setlength{\parskip}{6pt plus 2pt minus 1pt}}
}{% if KOMA class
  \KOMAoptions{parskip=half}}
\makeatother
\usepackage{xcolor}
\setlength{\emergencystretch}{3em} % prevent overfull lines
\setcounter{secnumdepth}{5}
% Make \paragraph and \subparagraph free-standing
\ifx\paragraph\undefined\else
  \let\oldparagraph\paragraph
  \renewcommand{\paragraph}[1]{\oldparagraph{#1}\mbox{}}
\fi
\ifx\subparagraph\undefined\else
  \let\oldsubparagraph\subparagraph
  \renewcommand{\subparagraph}[1]{\oldsubparagraph{#1}\mbox{}}
\fi

\usepackage{color}
\usepackage{fancyvrb}
\newcommand{\VerbBar}{|}
\newcommand{\VERB}{\Verb[commandchars=\\\{\}]}
\DefineVerbatimEnvironment{Highlighting}{Verbatim}{commandchars=\\\{\}}
% Add ',fontsize=\small' for more characters per line
\usepackage{framed}
\definecolor{shadecolor}{RGB}{241,243,245}
\newenvironment{Shaded}{\begin{snugshade}}{\end{snugshade}}
\newcommand{\AlertTok}[1]{\textcolor[rgb]{0.68,0.00,0.00}{#1}}
\newcommand{\AnnotationTok}[1]{\textcolor[rgb]{0.37,0.37,0.37}{#1}}
\newcommand{\AttributeTok}[1]{\textcolor[rgb]{0.40,0.45,0.13}{#1}}
\newcommand{\BaseNTok}[1]{\textcolor[rgb]{0.68,0.00,0.00}{#1}}
\newcommand{\BuiltInTok}[1]{\textcolor[rgb]{0.00,0.23,0.31}{#1}}
\newcommand{\CharTok}[1]{\textcolor[rgb]{0.13,0.47,0.30}{#1}}
\newcommand{\CommentTok}[1]{\textcolor[rgb]{0.37,0.37,0.37}{#1}}
\newcommand{\CommentVarTok}[1]{\textcolor[rgb]{0.37,0.37,0.37}{\textit{#1}}}
\newcommand{\ConstantTok}[1]{\textcolor[rgb]{0.56,0.35,0.01}{#1}}
\newcommand{\ControlFlowTok}[1]{\textcolor[rgb]{0.00,0.23,0.31}{#1}}
\newcommand{\DataTypeTok}[1]{\textcolor[rgb]{0.68,0.00,0.00}{#1}}
\newcommand{\DecValTok}[1]{\textcolor[rgb]{0.68,0.00,0.00}{#1}}
\newcommand{\DocumentationTok}[1]{\textcolor[rgb]{0.37,0.37,0.37}{\textit{#1}}}
\newcommand{\ErrorTok}[1]{\textcolor[rgb]{0.68,0.00,0.00}{#1}}
\newcommand{\ExtensionTok}[1]{\textcolor[rgb]{0.00,0.23,0.31}{#1}}
\newcommand{\FloatTok}[1]{\textcolor[rgb]{0.68,0.00,0.00}{#1}}
\newcommand{\FunctionTok}[1]{\textcolor[rgb]{0.28,0.35,0.67}{#1}}
\newcommand{\ImportTok}[1]{\textcolor[rgb]{0.00,0.46,0.62}{#1}}
\newcommand{\InformationTok}[1]{\textcolor[rgb]{0.37,0.37,0.37}{#1}}
\newcommand{\KeywordTok}[1]{\textcolor[rgb]{0.00,0.23,0.31}{#1}}
\newcommand{\NormalTok}[1]{\textcolor[rgb]{0.00,0.23,0.31}{#1}}
\newcommand{\OperatorTok}[1]{\textcolor[rgb]{0.37,0.37,0.37}{#1}}
\newcommand{\OtherTok}[1]{\textcolor[rgb]{0.00,0.23,0.31}{#1}}
\newcommand{\PreprocessorTok}[1]{\textcolor[rgb]{0.68,0.00,0.00}{#1}}
\newcommand{\RegionMarkerTok}[1]{\textcolor[rgb]{0.00,0.23,0.31}{#1}}
\newcommand{\SpecialCharTok}[1]{\textcolor[rgb]{0.37,0.37,0.37}{#1}}
\newcommand{\SpecialStringTok}[1]{\textcolor[rgb]{0.13,0.47,0.30}{#1}}
\newcommand{\StringTok}[1]{\textcolor[rgb]{0.13,0.47,0.30}{#1}}
\newcommand{\VariableTok}[1]{\textcolor[rgb]{0.07,0.07,0.07}{#1}}
\newcommand{\VerbatimStringTok}[1]{\textcolor[rgb]{0.13,0.47,0.30}{#1}}
\newcommand{\WarningTok}[1]{\textcolor[rgb]{0.37,0.37,0.37}{\textit{#1}}}

\providecommand{\tightlist}{%
  \setlength{\itemsep}{0pt}\setlength{\parskip}{0pt}}\usepackage{longtable,booktabs,array}
\usepackage{calc} % for calculating minipage widths
% Correct order of tables after \paragraph or \subparagraph
\usepackage{etoolbox}
\makeatletter
\patchcmd\longtable{\par}{\if@noskipsec\mbox{}\fi\par}{}{}
\makeatother
% Allow footnotes in longtable head/foot
\IfFileExists{footnotehyper.sty}{\usepackage{footnotehyper}}{\usepackage{footnote}}
\makesavenoteenv{longtable}
\usepackage{graphicx}
\makeatletter
\def\maxwidth{\ifdim\Gin@nat@width>\linewidth\linewidth\else\Gin@nat@width\fi}
\def\maxheight{\ifdim\Gin@nat@height>\textheight\textheight\else\Gin@nat@height\fi}
\makeatother
% Scale images if necessary, so that they will not overflow the page
% margins by default, and it is still possible to overwrite the defaults
% using explicit options in \includegraphics[width, height, ...]{}
\setkeys{Gin}{width=\maxwidth,height=\maxheight,keepaspectratio}
% Set default figure placement to htbp
\makeatletter
\def\fps@figure{htbp}
\makeatother
% definitions for citeproc citations
\NewDocumentCommand\citeproctext{}{}
\NewDocumentCommand\citeproc{mm}{%
  \begingroup\def\citeproctext{#2}\cite{#1}\endgroup}
\makeatletter
 % allow citations to break across lines
 \let\@cite@ofmt\@firstofone
 % avoid brackets around text for \cite:
 \def\@biblabel#1{}
 \def\@cite#1#2{{#1\if@tempswa , #2\fi}}
\makeatother
\newlength{\cslhangindent}
\setlength{\cslhangindent}{1.5em}
\newlength{\csllabelwidth}
\setlength{\csllabelwidth}{3em}
\newenvironment{CSLReferences}[2] % #1 hanging-indent, #2 entry-spacing
 {\begin{list}{}{%
  \setlength{\itemindent}{0pt}
  \setlength{\leftmargin}{0pt}
  \setlength{\parsep}{0pt}
  % turn on hanging indent if param 1 is 1
  \ifodd #1
   \setlength{\leftmargin}{\cslhangindent}
   \setlength{\itemindent}{-1\cslhangindent}
  \fi
  % set entry spacing
  \setlength{\itemsep}{#2\baselineskip}}}
 {\end{list}}
\usepackage{calc}
\newcommand{\CSLBlock}[1]{\hfill\break\parbox[t]{\linewidth}{\strut\ignorespaces#1\strut}}
\newcommand{\CSLLeftMargin}[1]{\parbox[t]{\csllabelwidth}{\strut#1\strut}}
\newcommand{\CSLRightInline}[1]{\parbox[t]{\linewidth - \csllabelwidth}{\strut#1\strut}}
\newcommand{\CSLIndent}[1]{\hspace{\cslhangindent}#1}

%%%% Standard Packages

\usepackage{graphicx}%
\usepackage{multirow}%
\usepackage{amsmath,amssymb,amsfonts}%
\usepackage{amsthm}%
\usepackage{mathrsfs}%
\usepackage[title]{appendix}%
\usepackage{xcolor}%
\usepackage{textcomp}%
\usepackage{manyfoot}%
\usepackage{booktabs}%
\usepackage{algorithm}%
\usepackage{algorithmicx}%
\usepackage{algpseudocode}%
\usepackage{listings}%

%%%%

\raggedbottom
\usepackage{float}
\usepackage{tabularray}
\usepackage[normalem]{ulem}
\usepackage{graphicx}
\UseTblrLibrary{booktabs}
\UseTblrLibrary{siunitx}
\NewTableCommand{\tinytableDefineColor}[3]{\definecolor{#1}{#2}{#3}}
\newcommand{\tinytableTabularrayUnderline}[1]{\underline{#1}}
\newcommand{\tinytableTabularrayStrikeout}[1]{\sout{#1}}
\usepackage{multicol}
\makeatletter
\@ifpackageloaded{caption}{}{\usepackage{caption}}
\AtBeginDocument{%
\ifdefined\contentsname
  \renewcommand*\contentsname{Table of contents}
\else
  \newcommand\contentsname{Table of contents}
\fi
\ifdefined\listfigurename
  \renewcommand*\listfigurename{List of Figures}
\else
  \newcommand\listfigurename{List of Figures}
\fi
\ifdefined\listtablename
  \renewcommand*\listtablename{List of Tables}
\else
  \newcommand\listtablename{List of Tables}
\fi
\ifdefined\figurename
  \renewcommand*\figurename{Figure}
\else
  \newcommand\figurename{Figure}
\fi
\ifdefined\tablename
  \renewcommand*\tablename{Table}
\else
  \newcommand\tablename{Table}
\fi
}
\@ifpackageloaded{float}{}{\usepackage{float}}
\floatstyle{ruled}
\@ifundefined{c@chapter}{\newfloat{codelisting}{h}{lop}}{\newfloat{codelisting}{h}{lop}[chapter]}
\floatname{codelisting}{Listing}
\newcommand*\listoflistings{\listof{codelisting}{List of Listings}}
\makeatother
\makeatletter
\makeatother
\makeatletter
\@ifpackageloaded{caption}{}{\usepackage{caption}}
\@ifpackageloaded{subcaption}{}{\usepackage{subcaption}}
\makeatother
\ifLuaTeX
  \usepackage{selnolig}  % disable illegal ligatures
\fi
\usepackage{bookmark}

\IfFileExists{xurl.sty}{\usepackage{xurl}}{} % add URL line breaks if available
\urlstyle{same} % disable monospaced font for URLs
\hypersetup{
  pdftitle={Article Template},
  pdfauthor={Author One; Author Two},
  pdfkeywords={template, demo},
  colorlinks=true,
  linkcolor={blue},
  filecolor={Maroon},
  citecolor={Blue},
  urlcolor={Blue},
  pdfcreator={LaTeX via pandoc}}

\title[Article Template]{Article Template}

% author setup
\author[1,2]{\fnm{Author} \sur{One}}\equalcont{These authors contributed equally to this work.}\author*[1]{\fnm{Author} \sur{Two}}\email{corresponding@email.com}\equalcont{These authors contributed equally to this work.}
% affil setup
\affil[1]{\orgdiv{Social Sciences}, \orgname{Humboldt
University}, \orgaddress{\street{street}, \city{Berlin}, \postcode{12345}}}
\affil[2]{\orgdiv{IPI}, \orgname{WZB}, \orgaddress{\street{here}, \city{Berlin}, \postcode{12345}}}

% abstract 

\abstract{The abstract serves both as a general introduction to the
topic and as a brief, non-technical summary of the main results and
their implications. Authors are advised to check the author instructions
for the journal they are submitting to for word limits and if structural
elements like subheadings, citations, or equations are permitted.}

% keywords
\keywords{template,  demo}

\begin{document}
\maketitle

\newpage

\section{Introduction}\label{sec-intro}

Your question. why we care

Your findings and the contribution

Roadmap

\section{Motivation: Theory, Literature}\label{sec-motivation}

\subsection{Cross Referencing}\label{cross-referencing}

\begin{itemize}
\tightlist
\item
  To reference a figure with example label ``plot'', use
  \texttt{@fig-plot}. Figure~\ref{fig-1}
\item
  Analogously, to reference a table with example label ``data'', use
  \texttt{@tbl-data}. Table~\ref{tbl-simple}.
\item
  To reference a section, such as the Introduction
  (Section~\ref{sec-intro}), use \texttt{@sec-intro}.
\end{itemize}

For complete information on cross referencing with Quarto, see
\url{https://quarto.org/docs/authoring/cross-references.html}.

\subsection{Citations}\label{citations}

\begin{itemize}
\tightlist
\item
  For a citation in parentheses use \texttt{{[}@greenwade93{]}} and for
  a text citation: \texttt{@greenwade93}.
\item
  These render as (Greenwade 1993) and Greenwade (1993)
\end{itemize}

Multiple citations can be given as
\texttt{{[}@greenwade93;@knuth1984texbook{]}}, producing (Knuth and
Bibby 1984; Greenwade 1993)

\newpage

\section{Tables and Figures}\label{tables-and-figures}

These can be produced on the fly or imported.

\subsection{A sample figure}\label{a-sample-figure}

Produced on the fly. See Figure~\ref{fig-1}.

\begin{Shaded}
\begin{Highlighting}[]
\FunctionTok{read\_rds}\NormalTok{(}\StringTok{"data/data.rds"}\NormalTok{) }\SpecialCharTok{|\textgreater{}}
  \FunctionTok{ggplot}\NormalTok{(}\FunctionTok{aes}\NormalTok{(X, Y)) }\SpecialCharTok{+} \FunctionTok{geom\_point}\NormalTok{()}
\end{Highlighting}
\end{Shaded}

\begin{figure}[H]

\centering{

\captionsetup{labelsep=none}\includegraphics{template_files/figure-pdf/fig-1-1.pdf}

}

\caption{\label{fig-1}}

\end{figure}%

\newpage

\section{A sample summary stats
table}\label{a-sample-summary-stats-table}

Produced on the fly. See Table~\ref{tbl-simple}.

\begin{Shaded}
\begin{Highlighting}[]
\FunctionTok{read\_rds}\NormalTok{(}\StringTok{"data/data.rds"}\NormalTok{) }\SpecialCharTok{|\textgreater{}}
  \FunctionTok{summarize}\NormalTok{(}\StringTok{\textasciigrave{}}\AttributeTok{mean Y}\StringTok{\textasciigrave{}} \OtherTok{=} \FunctionTok{mean}\NormalTok{(Y), }
            \StringTok{\textasciigrave{}}\AttributeTok{mean X}\StringTok{\textasciigrave{}} \OtherTok{=} \FunctionTok{mean}\NormalTok{(X), }
            \AttributeTok{correlation =} \FunctionTok{cor}\NormalTok{(X, Y)) }\SpecialCharTok{|\textgreater{}}
  \FunctionTok{kable}\NormalTok{(}\AttributeTok{digits =} \DecValTok{2}\NormalTok{, }\AttributeTok{caption =} \StringTok{"A sample table"}\NormalTok{)}
\end{Highlighting}
\end{Shaded}

\begin{longtable}[]{@{}rrr@{}}

\caption{\label{tbl-simple}A sample table}

\tabularnewline

\toprule\noalign{}
mean Y & mean X & correlation \\
\midrule\noalign{}
\endhead
\bottomrule\noalign{}
\endlastfoot
-0.03 & 0.02 & 0.69 \\

\end{longtable}

\newpage

\section{A sample analysis table}\label{a-sample-analysis-table}

Produced on the fly.

\begin{table}

\caption{\label{tbl-ols}}

\centering{

\captionsetup{labelsep=none}

\begin{Shaded}
\begin{Highlighting}[]
\FunctionTok{list}\NormalTok{(}
  \FunctionTok{lm\_robust}\NormalTok{(Y}\SpecialCharTok{\textasciitilde{}}\DecValTok{1}\NormalTok{, }\AttributeTok{data =} \FunctionTok{read\_rds}\NormalTok{(}\StringTok{"data/data.rds"}\NormalTok{)),}
  \FunctionTok{lm\_robust}\NormalTok{(Y}\SpecialCharTok{\textasciitilde{}}\NormalTok{X, }\AttributeTok{data =} \FunctionTok{read\_rds}\NormalTok{(}\StringTok{"data/data.rds"}\NormalTok{)))}\SpecialCharTok{|\textgreater{}}
  \FunctionTok{modelsummary}\NormalTok{()}
\end{Highlighting}
\end{Shaded}

\centering
\begin{tblr}[         %% tabularray outer open
]                     %% tabularray outer close
{                     %% tabularray inner open
colspec={Q[]Q[]Q[]},
column{1}={halign=l,},
column{2}={halign=c,},
column{3}={halign=c,},
hline{6}={1,2,3}{solid, 0.05em, black},
}                     %% tabularray inner close
\toprule
& (1) & (2) \\ \midrule %% TinyTableHeader
(Intercept) & \num{-0.033}  & \num{-0.050}  \\
& (\num{0.143}) & (\num{0.104}) \\
X           &                & \num{0.941}   \\
&                & (\num{0.083}) \\
Num.Obs.    & \num{100}     & \num{100}     \\
R2          & \num{0.000}   & \num{0.478}   \\
R2 Adj.     & \num{0.000}   & \num{0.473}   \\
AIC         &                & \num{295.7}   \\
BIC         &                & \num{303.6}   \\
RMSE        &                & \num{1.03}    \\
\bottomrule
\end{tblr}

}

\end{table}%

\newpage

\section{References}\label{references}

\phantomsection\label{refs}
\begin{CSLReferences}{1}{1}
\bibitem[\citeproctext]{ref-greenwade93}
Greenwade GD (1993) The comprehensive {TeX} {A}rchive {N}etwork
({CTAN}). TUGBoat 14:342--351

\bibitem[\citeproctext]{ref-knuth1984texbook}
Knuth DE, Bibby D (1984) The TeXbook. Addison-Wesley Reading

\end{CSLReferences}



\end{document}
